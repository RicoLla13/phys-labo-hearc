\documentclass[12pt,a4paper]{article}

\usepackage[left=3cm, right=3cm, top=2.5cm, bottom=2.5cm]{geometry}
\usepackage{setspace}
\usepackage{amsmath}
\usepackage{tikz}
\usepackage{pgfplotstable}
\usepackage{titlesec}
\usepackage{bm}
\usepackage{tcolorbox}
\tcbuselibrary{skins}
\usepackage{empheq}
\usepackage{booktabs}
\usepackage{caption}
\usepackage{hyperref}
\usepackage{fancyhdr}
\hypersetup{
    colorlinks=true,
    linkcolor=black,
    filecolor=magenta,      
    urlcolor=cyan,
    pdfpagemode=FullScreen,
    }
\usepackage{graphicx}
\graphicspath{ {./images/} }


\titleformat{\section}{\Large\bfseries}{\thesection}{1em}{}
\titleformat{\subsection}{\large\bfseries}{\thesubsection}{1em}{}


\renewcommand{\contentsname}{Table des Matières}
%\renewcommand{\baselinestretch}{1.5}

\title{Étude des équations balistiques}
\author{Liviu Arsenescu, Cătălin Bozan}
\date{date}

\newtcbox{\mymath}[1][]{%
    nobeforeafter,
    math upper,
    tcbox raise base,
    enhanced,
    colframe=black,
    colback=white,
    boxrule=1pt,
    drop shadow={
        shadow xshift=3pt,
        shadow yshift=-3pt,
        opacity=1
    },
    #1
}

\pagestyle{fancy}
\fancyhf{}
\rhead{\includegraphics[width=4cm]{hearclogo.png}}
\lhead{\thepage}
\setlength{\headsep}{30pt}

\begin{document}
    \pagenumbering{gobble}
    \begin{titlepage}
        \begin{center}
            \vspace*{\fill}
            \Huge \textbf{Étude des équations de la balistique :} \\
            \Huge \textbf{Équations du mouvement dans un champ gravitationnel} \\
            \Large Rapport du Laboratoire \\
            \begin{figure}[h]
                \centering
                \includegraphics[width=7cm]{hearclogo.png}
            \end{figure}
            \vspace{\fill}
            \Large Liviu Arsenescu, Cătălin Bozan \\
            19.03.2024

            \vspace*{\fill}
        \end{center}
    \end{titlepage}

    \thispagestyle{empty}
    \tableofcontents
    \newpage

    \pagenumbering{arabic}
    \section{Description de l'expérience}
    \subsection{Buts}
    \begin{itemize}
        \item Vérifier les équations de la balistique
        \item Obtenit la vitesse à laquelle le canon tire le boulet
        \item Pédire la hauteur de l'impact de la bille contre un mur avec un tir oblique
    \end{itemize}

    \subsection{Éléments théoriques}
    \subsubsection{Les différentes grandeurs physiques rencontrées}
    \begin{minipage}{0.6\linewidth}
        \begin{itemize}
            \item \textbf{h} - hauteur initiale 
            \item \textbf{d} - distance entre le canon et le point d'impact 
            \item \textbf{H} - hauteur d'impact
            \item $\bm{\theta}$ - angle de tir 
            \item $\bm{v_0}$ - vitesse de sortie du canon
        \end{itemize}
    \end{minipage}%
    \hfill
    \begin{minipage}{0.4\linewidth}
        \begin{itemize}
            \item[-] $\bm{[h]=cm}$
            \item[-] $\bm{[d]=cm}$
            \item[-] $\bm{[H]=cm}$
            \item[-] $\bm{[\theta]=deg}$
            \item[-] $\bm{[v_0]=cms^{-1}}$
        \end{itemize}   
    \end{minipage}
    \subsubsection{Les équations de la balistique}
    La théorie nous donne les équations suivantes pour le mouvement d'un corps en configuration balistique:
    \begin{itemize}
        \item Pour l'axe x:
        \begin{align*}
            x(t)&=v_{x0}+x_0 \\
            v_x(t)&=v_{x0}
        \end{align*}
        \item Pour l'axe y
        \begin{align*}
            y&=-\frac{1}{2}gt^2+v_{y0}t+y_0 \\
            v_y(t)&=-gt+v_{y0}
        \end{align*}
    \end{itemize}
    On peut utiliser ces équations pour construire deux systèmes contenant un mouvement balistique:
    \paragraph{Tir horizontal:}

    \subsection{Principe de l'expérience}
    \subsection{Schéma et montage de l’expérience}
    \subsection{Déroulement de l'expérience}

    \section{Mesures}

    \section{Analyse des mesures et résultats}

    \section{Synthèse et conclusion}
\end{document}

