\documentclass[12pt,a4paper]{article}

\usepackage[left=2cm, right=2cm, top=2cm, bottom=2cm]{geometry}
\usepackage{setspace}
\usepackage{amsmath}
\usepackage{tikz}
\usepackage{pgfplotstable}
\usepackage{titlesec}

\titleformat{\section}{\Large\bfseries}{\thesection}{1em}{}
\titleformat{\subsection}{\large\bfseries}{\thesubsection}{1em}{}

\renewcommand{\contentsname}{Table des Matières}
%\renewcommand{\baselinestretch}{1.5}

\title{Étude du Pendule Simple : Analyse des Oscillations Harmoniques}
\author{Cătălin Bozan, Liviu Arsenescu}
\date{05.03.2024}

\begin{document}
    \pagenumbering{gobble}
    \begin{titlepage}
        \begin{center}
            \vspace*{\fill}
            \Huge \textbf{Étude du Pendule Simple :} \\
            \Huge \textbf{Analyse des Oscillations Harmoniques} \\
            \Large Rapport du Laboratoire \\
            \vspace{\fill}
            \Large Cătălin Bozan, Liviu Arsenescu \\
            05.03.2024

            \vspace*{\fill}
        \end{center}
    \end{titlepage}
    \newpage

    \pagenumbering{arabic}
    \section{Objectifs du laboratoire}
    \begin{itemize}
        \item Démontrer expérimentalement le fait que la période ne dépend pas de la masse, lorsque la longueur est constante
        \item Vérifier la formule de la période d'un pendule
        \item Trouver l'accélération gravitationnelle de la terre
    \end{itemize}
    \newpage

    \section{Éléments théoriques}
    \newpage

    \section{Manipulation}
    \newpage

    \section{Mesures}
    \newpage

    \section{Analyse des mesures et résultats}
    \newpage

    \section{Synthèse et conclusion}
\end{document}

