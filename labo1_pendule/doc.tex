\documentclass[12pt,a4paper]{article}

\usepackage[left=3cm, right=3cm, top=2.5cm, bottom=2.5cm]{geometry}
\usepackage{setspace}
\usepackage{amsmath}
\usepackage{tikz}
\usepackage{pgfplotstable}
\usepackage{titlesec}
\usepackage{bm}
\usepackage{tcolorbox}
\tcbuselibrary{skins}
\usepackage{empheq}
\usepackage{booktabs}
\usepackage{caption}

\titleformat{\section}{\Large\bfseries}{\thesection}{1em}{}
\titleformat{\subsection}{\large\bfseries}{\thesubsection}{1em}{}

\renewcommand{\contentsname}{Table des Matières}
%\renewcommand{\baselinestretch}{1.5}

\title{Étude du Pendule Simple : Analyse des Oscillations Harmoniques}
\author{Liviu Arsenescu, Cătălin Bozan}
\date{05.03.2024}

\newtcbox{\mymath}[1][]{%
    nobeforeafter,
    math upper,
    tcbox raise base,
    enhanced,
    colframe=black,
    colback=white,
    boxrule=1pt,
    drop shadow={
        shadow xshift=3pt,
        shadow yshift=-3pt,
        opacity=1
    },
    #1
}

\begin{document}
    \pagenumbering{gobble}
    \begin{titlepage}
        \begin{center}
            \vspace*{\fill}
            \Huge \textbf{Étude du Pendule Simple :} \\
            \Huge \textbf{Analyse des Oscillations Harmoniques} \\
            \Large Rapport du Laboratoire \\
            \vspace{\fill}
            \Large Liviu Arsenescu, Cătălin Bozan \\
            05.03.2024

            \vspace*{\fill}
        \end{center}
    \end{titlepage}
    \newpage

    \pagenumbering{arabic}
    \section{Objectifs du laboratoire}
    \begin{itemize}
        \item Démontrer expérimentalement le fait que la période ne dépend pas de la masse
        \item Vérifier la formule de la période d'un pendule
        \item Trouver l'accélération gravitationnelle de la terre
    \end{itemize}

    \section{Éléments théoriques}
    \subsection{Les différentes quantités rencontrées}
    \begin{itemize}
        \item $\bm{\theta}$ - l'angle entre la verticale et le pendule
        \item \textbf{L} - la longueur du fil
        \item \textbf{T} - la période du pendule
        \item \textbf{m} - masse d'objet
        \item \textbf{s} - la position de la masse
        \item \textbf{g} - l'accélération gravitationnelle
    \end{itemize}
    \subsection{La formule fondamentale du pendule}
    La position de la masse suspendue est calculée à l'aide de la formule suivante: 
    \begin{equation*}
        s=L\theta,
    \end{equation*}
    La deuxième loi de Newton ($\sum_{i=1}^{n} F_i=ma$) selon l'axe tangentiele s'écrit:
    \begin{equation*}
        -mgsin(\theta) = m\frac{d^2s}{dt^2}
    \end{equation*}
    où $-mgsin(\theta)$ est l'équation obtenue pour la seule force agissant sur l'objet (P - poids), et $\frac{d^2s}{dt^2}$ est l'accélération totale du système, obtenue en dérivant deux fois la position. \\
    On sait que $\frac{d^2s}{dt^2} = L\frac{d^2\theta}{dt^2}$ (L - constante). Alors:
    \begin{align*}
        -mgsin(\theta)&=mL\frac{d^2\theta}{dt^2} \\
        -\frac{g}{L}sin(\theta)&=\frac{d^2\theta}{dt^2} \\
        \frac{d^2\theta}{dt^2}+\frac{g}{L}sin(\theta)&=0
    \end{align*}
    Pour notre expérience, nous n'utilisons que de petits angles, ce qui nous permet de faire l'approximation suivante: $sin(\theta) \approx \theta$. Le pendule simple devient alors un système oscillatoire harmonique, décrit par l'équation suivante:
    \begin{empheq}[box={\mymath}]{equation*}
        \frac{d^2\theta}{dt^2}+\frac{g}{L}\theta=0
    \end{empheq}
    En comparant cette équation avec l'équation du mouvement oscillatoire harmonique ($\frac{d^2x}{dt^2}+\omega^2x(t)=0$), nous obtenons:
    \begin{center}
        \vspace{-\baselineskip}
        \begin{minipage}{0.2\linewidth}
            \begin{empheq}[box={\mymath}]{equation*}
                \omega=\sqrt{\frac{g}{L}}
            \end{equation*}
        \end{minipage}%
        \begin{minipage}{0.1\linewidth}
            \begin{center}
                et
            \end{center}
        \end{minipage}%
        \begin{minipage}{0.2\linewidth}
            \begin{empheq}[box={\mymath}]{equation*}
                T=\frac{2\pi}{\omega}=2\pi\sqrt{\frac{L}{g}}
            \end{equation*}
        \end{minipage}
    \end{center}
    Cette approche mathématique permet de tirer les conclusions suivantes:
    \begin{itemize}
        \item La période ne dépends pas de la masse de l'objet
        \item La période est proportionnelle à la racine carrée de la longueur
        \item Nous pouvons estimer expérimentalement la norme d'accélération gravitationnelle à l'aide de la formule: 
        \begin{equation*}
            g=\frac{4\pi^2L}{T^2}
        \end{equation*}
    \end{itemize}

    \section{Manipulation}
    \subsection{Matériel}
    \begin{itemize}
        \item Socle
        \item Tiges
        \item Noix-double
        \item Tige-pince
        \item Diverses masses
        \item Fil non élastique
        \item Ruban métrique
        \item Camera vidéo
        \item Ordinateur
    \end{itemize}
    \subsection{Configuration}
    La tige est fixée verticalement au socle. En utilisant la noix-double, on fixe horizontalement

    \section{Mesures}
    \subsection{Manière de mesurer}
    Pour reduire l'incertitude, on n'a pas utilisé de chronomètre pour mesurer le temps, mais on a pris une vidéo avec une caméra de 60 cadres par seconde. Ensuite, on a analysé la vidéo à l'aide d'un logiciel de montage, pour mieux voir les cadres du début et de la fin. \\
    Bien que les instructions du laboratoire indiquent que nous devrions enregistrer 10 périodes, pendant le laboratoire on a eu une erreur avec les vidéos, ce qui nous a permis d'enregistrer seulement 7 périodes. Cela n'a pas eu un grand impact sur l'incertitude. \\
    Pour le reste des mesures(longueur, masse), on a suivi les instructions de laboratoire.
    \section{Analyse des mesures et résultats}
    \subsection{Période d'oscillation \textit{T} en fonction de la masse \textit{m}}
    Pour cette expérience, nous avons une longueur fixe pour le fil et on a augmenté la masse. \\
    Les valeurs suivantes restent constantes pour chaque mesure:
    \begin{itemize}
        \item Longueur: $L=80.0cm$
        \item Incertitude de la longueur: $\Delta L=\pm 0.4cm$
        \item Incertitude de la masse: $\Delta m=\pm 0.01g$
        \item Incertitude de 7 périodes: $\Delta 7T=\pm 0.05s \Rightarrow \Delta T=\pm 0.008s$
    \end{itemize}
    Le reste des valeurs peut être représenté dans un tableau:
    \begin{table}[htbp]
        \centering
        \begin{minipage}{0.4\textwidth}
            \begin{tabular}{c|c|c|c}
                \textbf{No.} & \textbf{m(g)} & \textbf{7T(s)} & \textbf{T(s)} \\
                \toprule
                1 & 19.96 & 12.52 & 1.788 \\
                2 & 49.76 & 12.60 & 1.800 \\
                3 & 99.63 & 12.55 & 1.792 \\
                4 & 199.69 & 12.57 & 1.795 \\
                5 & 499.46 & 12.52 & 1.788 \\
            \end{tabular}
        \end{minipage}%
        \begin{minipage}{0.6\textwidth}
            \centering
            \fbox{%
                \begin{tabular}{rcl}
                    Symb. & & description \\
                    \toprule
                    m(g) & - & masse en grammes \\
                    7T(s) & - & mesure de 7 périodes en secondes \\
                    T(s) & - & période en secondes \\
                \end{tabular}%
            }
        \end{minipage}
    \end{table} \\
    On peut utiliser cet ensemble de données pour faire un graphique:
    \section{Synthèse et conclusion}
\end{document}

