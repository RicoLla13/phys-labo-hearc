\documentclass[12pt,a4paper]{article}

\usepackage[left=3cm, right=3cm, top=2.5cm, bottom=2.5cm, headheight=38.40865pt]{geometry} % Adjusted headheight
\usepackage{setspace}
\usepackage{amsmath}
\usepackage{tikz}
\usepackage{pgfplotstable}
\usepackage{titlesec}
\usepackage{bm}
\usepackage{tcolorbox}
\tcbuselibrary{skins}
\usepackage{empheq}
\usepackage{booktabs}
\usepackage{caption}
\usepackage{hyperref}
\usepackage{fancyhdr}
\usepackage{float}
\usepackage{silence}
\usepackage{multirow}
\WarningFilter{caption}{The option `hypcap=true' will be ignored}
\WarningFilter{latex}{Underfull \hbox}
\hypersetup{
    colorlinks=true,
    linkcolor=black,
    filecolor=magenta,      
    urlcolor=cyan,
    pdfpagemode=FullScreen,
    }
\usepackage{graphicx}
\graphicspath{ {./images/} }

\pgfplotsset{compat=1.18}

\titleformat{\section}{\Large\bfseries}{\thesection}{1em}{}
\titleformat{\subsection}{\large\bfseries}{\thesubsection}{1em}{}

\renewcommand{\contentsname}{Table des Matières}
\renewcommand{\tablename}{Tableau}

\title{Interférences et diffraction de la lumière}
\author{Liviu Arsenescu, Cătălin Bozan}
\date{28.05.2024}

\newtcbox{\mymath}[1][]{%
    nobeforeafter,
    tcbox raise base,
    enhanced,
    colframe=black,
    colback=white,
    boxrule=1pt,
    drop shadow={
        xshift=3pt, % Removed 'shadow' prefix
        yshift=-3pt, % Removed 'shadow' prefix
        opacity=1
    },
    #1
}

\pagestyle{fancy}
\fancyhf{}
\rhead{\includegraphics[width=4cm]{hearclogo.png}}
\lhead{\thepage}
\setlength{\headsep}{30pt}

\begin{document}
    \pagenumbering{gobble}
    \begin{titlepage}
        \begin{center}
            \vspace*{\fill}
            \Huge \textbf{Interférences et diffraction de la lumière} \\
            \Large Rapport du Laboratoire \\
            \begin{figure}[h]
                \centering
                \includegraphics[width=7cm]{hearclogo.png}
            \end{figure}
            \vspace{\fill}
            \Large Liviu Arsenescu, Cătălin Bozan \\
            28.05.2024

            \vspace*{\fill}
        \end{center}
    \end{titlepage}

    \section{Buts}
    \begin{itemize}
        \item Étudier le caractère ondulatoire de la lumière
        \item Observer la diffraction de la lumière
        \item Observer les interférences
        \item Utiliser des objets ordinaires (cheveux) pour provoquer les deux effets
    \end{itemize}
    \section{Comment atteindre les buts}
    \begin{itemize}
        \item On fait passer un laser à travers une fente simple, puis on mesure les 8 premièrs couples de ($m$, $y$) pour enfin calculer la longueur d'onde de la lumière de laser
        \item On fait passer un laser à travers une fente double, puis on mesure les 8 premièrs couples de ($m$, $y$), et enfin, avec la longueur d'onde calculé avant, on calcule la distance $d$ entre les deux fentes
        \item On fait passer le laser à travers un cheveux, puis on calcule son diamètre avec les cinq premièrs couples de ($m$, $y$), et la longuer d'onde $\lambda_{exp}$ 
    \end{itemize}
    \section{Résultats}
    \begin{minipage}{\textwidth}
        \centering
        \begin{tabular}{c|c|c|c}
            \toprule
              & 1. Single Slit & 2. Double Slit & 3. Cheveux \\
            \toprule
            m & $y$ (mm)  & $y$ (mm) & $y$ (mm) \\
            \midrule
            1 & 2.2  & 1.4  & 5    \\
            2 & 4.5  & 3    & 10.5 \\
            3 & 7    & 4.3  & 16   \\ 
            4 & 10.4 & 5.9  & 21.3 \\
            5 & 11.6 & 7.4  & 27   \\
            6 & 14   & 9.1  &      \\
            7 & 16.4 & 10.4 &      \\
            8 & 18.9 & 12.3 &      \\
            \toprule
              & $\lambda$ (nm) & $d$ (mm) & $a$ ($\mu$m) \\
            \toprule
              & 688 & 0.25 & 69 \\
            \toprule
              & $\Delta \lambda$ (nm) & $\Delta d$ (mm) & $\Delta a$ ($\mu$m) \\
            \toprule
              & 63 & 0.03 & 7 \\
            \bottomrule
        \end{tabular}
    \end{minipage}
    \section{Conclusions}
    On peut constater que les résultats se situent dans les fourchettes attendues ($\lambda_{exp} \in (650 \pm 20) \textrm{ nm}$, $d \in (0.25 \pm 0.01) \textrm{ mm}$, $a \in [40; 100]$ \textrm{$\mu$m}). \\
    On peut donc conclure que les objectifs ont été atteints, et que les lois données par la théorie ont été démontrées avec succès.
\end{document}
