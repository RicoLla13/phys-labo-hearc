\documentclass[12pt,a4paper]{article}

\usepackage[left=3cm, right=3cm, top=2.5cm, bottom=2.5cm]{geometry}
\usepackage{setspace}
\usepackage{amsmath}
\usepackage{tikz}
\usepackage{pgfplotstable}
\usepackage{titlesec}
\usepackage{bm}
\usepackage{tcolorbox}
\tcbuselibrary{skins}
\usepackage{empheq}
\usepackage{booktabs}
\usepackage{caption}
\usepackage{hyperref}
\usepackage{fancyhdr}
\hypersetup{
    colorlinks=true,
    linkcolor=black,
    filecolor=magenta,      
    urlcolor=cyan,
    pdfpagemode=FullScreen,
    }
\usepackage{graphicx}
\graphicspath{ {./images/} }


\titleformat{\section}{\Large\bfseries}{\thesection}{1em}{}
\titleformat{\subsection}{\large\bfseries}{\thesubsection}{1em}{}


\renewcommand{\contentsname}{Table des Matières}
\renewcommand{\tablename}{Tableau }
%\renewcommand{\baselinestretch}{1.5}

\title{#Titlu#}
\author{Liviu Arsenescu, Cătălin Bozan}
\date{date}

\newtcbox{\mymath}[1][]{%
    nobeforeafter,
    math upper,
    tcbox raise base,
    enhanced,
    colframe=black,
    colback=white,
    boxrule=1pt,
    drop shadow={
        shadow xshift=3pt,
        shadow yshift=-3pt,
        opacity=1
    },
    #1
}

\pagestyle{fancy}
\fancyhf{}
\rhead{\includegraphics[width=4cm]{hearclogo.png}}
\lhead{\thepage}
\setlength{\headsep}{30pt}

\begin{document}
    \pagenumbering{gobble}
    \begin{titlepage}
        \begin{center}
            \vspace*{\fill}
            \Huge \textbf{#Titlu#} \\
            \Huge \textbf{#Subtitlu#} \\
            \Large Rapport du Laboratoire \\
            \begin{figure}[h]
                \centering
                \includegraphics[width=7cm]{hearclogo.png}
            \end{figure}
            \vspace{\fill}
            \Large Liviu Arsenescu, Cătălin Bozan \\
            09.04.2024

            \vspace*{\fill}
        \end{center}
    \end{titlepage}

    \thispagestyle{empty}
    \tableofcontents
    \newpage

    \pagenumbering{arabic}
    \section{Description de l'expérience}
    \subsection{Buts}
    \begin{itemize}
        \item 
    \end{itemize}

    \subsection{Éléments théoriques}
    \subsubsection{Les différentes grandeurs physiques rencontrées}
    \begin{minipage}{0.6\linewidth}
        \begin{itemize}
            \item \textbf{ceva} - descriere
        \end{itemize}
    \end{minipage}%
    \hfill
    \begin{minipage}{0.4\linewidth}
        \begin{itemize}
            \item[-] $\bm{[ceva]=unitate}$
        \end{itemize}   
    \end{minipage}

    \subsection{Principe de l'expérience}
    L'expérience consiste en ces deux parties :
    \begin{enumerate}
        \item Avant de procéder à l'expérience proprement dite, on calcule la masse linéique de la corde à l'aide de différentes mesures liées au poids et à la corde, qu'on utilise ensuite, avec la force $F$, pour calculer la vitesse $v_{calc}$.
        \item On mesure les paires de ($n$, $f_n$) pour les douze premières harmoniques, puis on les utilise pour construire une régression linéaire, à partir de la pente de laquelle on extrait la vitesse $v_{exp}$.
    \end{enumerate}

    \subsection{Schéma et montage de l’expérience}
    \subsection{Déroulement de l'expérience}
    \subsubsection{Les mesures préalables}
    \begin{itemize}
        \item On prend une balance et on mesure la masse totale de la corde.
        \item On établit une longueur étalon, qui servira à calculer le coefficient d'allongement de la corde.
        \item Avec cette longueur initiale, on calcule la longueur de la corde après avoir attaché le même poids que on utilise plus tard pour mettre en place l'expérience.
    \end{itemize}
    \subsubsection{Le calcul des harmoniques}
    \begin{itemize}
        \item On met l'agitateur sur une tige fixée à la table de travail.
        \item En prenant une distance considérable, on met une poulie de la même manière.
        \item On attache et on place la corde entre l'agitateur et la poulie, et on la tend à l'aide d'un poids à l'extrémité libre.
        \item On mesure la distance résultante entre l'extrémité de l'agitateur et la poulie.
        \item On connecte l'agitateur à un générateur de courant alternatif réglable.
        \item En ajustant la fréquence et la tension (amplitude) du générateur, on génère les douze premières harmoniques du notre système.
    \end{itemize}
    \section{Mesures}
    \subsection{Mesures constantes :}
    \begin{itemize}
        \item $L=(1.79 \pm 0.05)$ m - longueur entre l'agitateur et la poulie
        \item $m=(199.6 \pm 0.1)$ g - poids pour la fixation de la corde
        \item $m_c=(10.3 \pm 0.1)$ g - masse de la corde
        \item $l'=(0.39 \pm 0.03)$ m - longueur de la corde avant l'étirement
        \item $l''=(0.43 \pm 0.03)$ m - longueur de la corde après l'étirement
    \end{itemize}
    \subsection{Tableau des mesures :}

    \begin{minipage}{0.4\textwidth}
        \centering
        \begin{tabular}{c|c|c}
            \toprule
            $n$ & $f_n$ (Hz)  & $\Delta f_n$ (Hz) \\
            \midrule
            1  & 6.37  & 0.02 \\
            2  & 13.18 & 0.02 \\
            3  & 19.68 & 0.02 \\
            4  & 26.21 & 0.02 \\
            5  & 32.51 & 0.02 \\
            6  & 39.51 & 0.02 \\
            7  & 45.71 & 0.02 \\
            8  & 52.81 & 0.02 \\
            9  & 59.21 & 0.02 \\
            10 & 65.71 & 0.02 \\
            11 & 72.51 & 0.02 \\
            12 & 78.71 & 0.02 \\
            \bottomrule
        \end{tabular}
        \captionof{table}{Les douze premières harmoniques}
    \end{minipage}%
    \hfill
    \begin{minipage}{0.6\textwidth}
        Légende :
        \begin{itemize}
            \item $n$ - nombre de l'harmonique
            \item $f_n$ - fréquence de l'harmonique $n$ (en Hz)
            \item $\Delta f_n$ - incertitude de la fréquence (en Hz)
        \end{itemize}
    \end{minipage}

    \newpage
    \section{Analyse des mesures et résultats}
    \subsection{Vitesse avec la masse linéique}
    Pour calculer la vitesse à l'aide de la force $F$ et la masse linéique $\mu$, on utilise la formule de la partie théorique :
    \begin{empheq}[box={\mymath}]{equation*}
        v_{calc} = \sqrt{\frac{F}{\mu}}
    \end{equation*}
    On obtient, donc :
    \begin{empheq}[box={\mymath}]{equation*}
        v_{calc} = (20 \pm 6) \textrm{ ms}^{-1}
    \end{equation*}
    \subsection{Vitesse avec les harmoniques}
    Pour construire la régression linéaire, on partira de la formule suivante déduite dans la partie théorique :
    \begin{empheq}[box={\mymath}]{equation*}
        f_n = \frac{v_{exp}}{2L} n
    \end{equation*}
    On constate que la valeur $f_n$ est une fonction de $n$, et on voit aussi que $\frac{v_{exp}}{2L}$ est la pente de la fonction, qu'on note par $a$ pour les calculs qui suivre. \\ 
    On obtient, donc :
    
    \pgfplotstableread[col sep=comma]{data/exp_data.csv}\expIdata
    \begin{tikzpicture}[scale=1]
        % Scatter plot
        \begin{axis}[
            xlabel={$n$},
            ylabel={$f_n$ (Hz)},
            legend pos=north east,
            legend style={at={(0.30,0.75)}, anchor=south},
            grid=both,
            width=1\textwidth,
            height=0.3\textheight,
            x tick label style={
                /pgf/number format/.cd,
                precision=0,
                fixed,
                fixed zerofill,
            },
            y tick label style={
                /pgf/number format/.cd,
                precision=2,
                fixed,
                fixed zerofill,
            },
            xmin=0,
            xmax=13,
            ymin=4,
            ymax=81,
        ]
            \addplot+[only marks, mark=x, error bars/.cd, x dir=both, x explicit, y dir=both, y explicit] table[x=n, x error=err_g, y=fn, y error=err_fn] {\expIdata};

            % Linear regression line
            \addplot [red, thick] table[
                y={create col/linear regression={y=fn}}
            ] {\expIdata};
            \xdef\slope{\pgfplotstableregressiona}
            \xdef\intercept{\pgfplotstableregressionb}

            % Add the equation of the line
            \addlegendentry{Données}
            \addlegendentry{Régr. Lin.: $y = \pgfmathprintnumber{\slope}x + \pgfmathprintnumber{\intercept}$}
        \end{axis}
    \end{tikzpicture}
    Après avoir extrait la vitesse et l'incertitude de la valeur 
de la pente, on a :
    \begin{empheq}[box={\mymath}]{equation*}
        v_{exp} = (23.6 \pm 0.8) \textrm{ ms}^{-1}
    \end{equation*}
    \subsection{Choix et calcul d'incertitudes}
    \subsubsection{Choix des incertitude :}
    \subsubsection{Calcul d'incertitudes}
    \subsection{Discussion des résultats :}
    \section{Synthèse et conclusion}
\end{document}

