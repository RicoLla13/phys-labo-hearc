\documentclass[12pt,a4paper]{article}

\usepackage[left=3cm, right=3cm, top=2.5cm, bottom=2.5cm]{geometry}
\usepackage{setspace}
\usepackage{amsmath}
\usepackage{tikz}
\usepackage{pgfplotstable}
\usepackage{titlesec}
\usepackage{bm}
\usepackage{tcolorbox}
\tcbuselibrary{skins}
\usepackage{empheq}
\usepackage{booktabs}
\usepackage{caption}
\usepackage{hyperref}
\usepackage{fancyhdr}
\hypersetup{
    colorlinks=true,
    linkcolor=black,
    filecolor=magenta,      
    urlcolor=cyan,
    pdfpagemode=FullScreen,
    }
\usepackage{graphicx}
\graphicspath{ {./images/} }


\titleformat{\section}{\Large\bfseries}{\thesection}{1em}{}
\titleformat{\subsection}{\large\bfseries}{\thesubsection}{1em}{}


\renewcommand{\contentsname}{Table des Matières}
%\renewcommand{\baselinestretch}{1.5}

\title{title}
\author{Liviu Arsenescu, Cătălin Bozan}
\date{date}

\newtcbox{\mymath}[1][]{%
    nobeforeafter,
    math upper,
    tcbox raise base,
    enhanced,
    colframe=black,
    colback=white,
    boxrule=1pt,
    drop shadow={
        shadow xshift=3pt,
        shadow yshift=-3pt,
        opacity=1
    },
    #1
}

\pagestyle{fancy}
\fancyhf{}
\rhead{\includegraphics[width=4cm]{hearclogo.png}}
\lhead{\thepage}
\setlength{\headsep}{30pt}

\begin{document}
    \pagenumbering{gobble}
    \begin{titlepage}
        \begin{center}
            \vspace*{\fill}
            \Huge \textbf{title :} \\
            \Huge \textbf{subtitle} \\
            \Large Rapport du Laboratoire \\
            \begin{figure}[h]
                \centering
                \includegraphics[width=7cm]{hearclogo.png}
            \end{figure}
            \vspace{\fill}
            \Large Liviu Arsenescu, Cătălin Bozan \\
            date

            \vspace*{\fill}
        \end{center}
    \end{titlepage}

    \thispagestyle{empty}
    \tableofcontents
    \newpage

    \pagenumbering{arabic}
    \section{Objectifs du laboratoire}
    \begin{itemize}
        \item 
    \end{itemize}

    \section{Éléments théoriques}
    \subsection{Les différentes quantités rencontrées}
    \begin{itemize}
        \item 
    \end{itemize}

    \section{Comment va-t-on atteindre les buts?}

    \section{Manipulation}
    \subsection{Matériel}
    \subsection{Configuration}

    \section{Mesures}
    \subsection{Manière de mesurer}

    \section{Analyse des mesures et résultats}

    \section{Synthèse et conclusion}
\end{document}

